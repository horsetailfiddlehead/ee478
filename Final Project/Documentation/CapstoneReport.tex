% !TEX TS-program = pdflatex
% !TEX encoding = UTF-8 Unicode

% This is a simple template for a LaTeX document using the "article" class.
% See "book", "report", "letter" for other types of document.

\documentclass[12pt]{article} % use larger type; default would be 10pt
\usepackage[utf8]{inputenc}   % set input encoding (not needed with XeLaTeX)

%%% PAGE DIMENSIONS
\usepackage{geometry}
\geometry{letterpaper}
\geometry{margin=1in} % 1in page margin

%%% COLOR AND GRAPHICS
\usepackage{color}
\usepackage{graphicx} % support the \includegraphics command and options

\usepackage{pslatex}
\definecolor{mygreen}{rgb}{0,0.6,0}
\definecolor{mygray}{rgb}{0.5,0.5,0.5}
\definecolor{mymauve}{rgb}{0.58,0,0.82}
\usepackage{listings} % For displaying source code
\lstset{ %
	language=C,                      % the language of the code
	backgroundcolor=\color{white},   % choose the background color; you must add \usepackage{color} or \usepackage{xcolor}
	basicstyle=\sffamily\fontsize{11}{13.2}\selectfont,        % the size of the fonts that are used for the code
	breakatwhitespace=false,         % sets if automatic breaks should only happen at whitespace
	breaklines=true,                 % sets automatic line breaking
	captionpos=t,                    % sets the caption-position to bottom
	commentstyle=\color{mygreen},    % comment style
	deletekeywords={...},            % if you want to delete keywords from the given language
	escapeinside={\%*}{*)},          % if you want to add LaTeX within your code
	extendedchars=true,              % lets you use non-ASCII characters; for 8-bits encodings only, does not work with UTF-8
	frame=single,                    % adds a frame around the code
	keepspaces=true,                 % keeps spaces in text, useful for keeping indentation of code (possibly needs columns=flexible)
	keywordstyle=\color{blue},       % keyword style
	morekeywords={*,...},            % if you want to add more keywords to the set
	numbers=left,                    % where to put the line-numbers; possible values are (none, left, right)
	numbersep=5pt,                   % how far the line-numbers are from the code
	numberstyle=\color{mygray},      % the style that is used for the line-numbers
	rulecolor=\color{black},         % if not set, the frame-color may be changed on line-breaks within not-black text (e.g. comments (green here))
	showspaces=false,                % show spaces everywhere adding particular underscores; it overrides 'showstringspaces'
	showstringspaces=false,          % underline spaces within strings only
	showtabs=false,                  % show tabs within strings adding particular underscores
	stepnumber=1,                    % the step between two line-numbers. If it's 1, each line will be numbered
	stringstyle=\color{mymauve},     % string literal style
	tabsize=2,                       % sets default tabsize to 2 spaces
	title=\lstname                   % show the filename of files included with \lstinputlisting; also try caption instead of title
}

% \usepackage[parfill]{parskip} % Activate to begin paragraphs with an empty line rather than an indent

%%% PACKAGES
\usepackage{booktabs} % for much better looking tables
\usepackage{array}    % for better arrays (eg matrices) in maths
\usepackage{paralist} % very flexible & customisable lists (eg. enumerate/itemize, etc.)
\usepackage{verbatim} % adds environment for commenting out blocks of text & for better verbatim
%\usepackage{subfig}   % make it possible to include more than one captioned figure/table in a single float
\usepackage{subcaption}
\usepackage{adjustbox}

%%% HEADERS & FOOTERS
%\usepackage{fancyhdr} % This should be set AFTER setting up the page geometry
%\pagestyle{fancy} % options: empty , plain , fancy
%\renewcommand{\headrulewidth}{0pt} % customise the layout...
%\lhead{}\chead{}\rhead{}
%\lfoot{}\cfoot{\thepage}\rfoot{}


%%% SECTION TITLE APPEARANCE
\usepackage{sectsty}
\sectionfont{\normalsize\bfseries\uppercase}
\subsectionfont{\normalsize\bfseries}
\subsubsectionfont{\normalsize\mdseries\itshape}

%%% ToC (table of contents) APPEARANCE
\usepackage[nottoc,notlof,notlot]{tocbibind} % Put the bibliography in the ToC
\usepackage[titles]{tocloft} % Alter the style of the Table of Contents
\renewcommand{\cftsecfont}{\rmfamily\mdseries\upshape}
\renewcommand{\cftsecpagefont}{\rmfamily\mdseries\upshape} % No bold!

%%% Title setup
\newcommand{\TitleFont}{\fontsize{16}{20}\selectfont\bfseries}
\newcommand{\AuthorFont}{\fontsize{14}{17}\selectfont}

%%% END Article customizations

%%% The "real" document content comes below...

\title{\TitleFont EE 478 Capstone Final Report \\ RFID Interaction Suite \vfill }
\author{\AuthorFont Alyanna Castillo \\ Patrick Ma \\ Ryan McDaniels}
\date{}

\begin{document}

%% Make title and ToC, start page numbering AFTER ToC
\maketitle
\thispagestyle{empty}
\pagebreak \tableofcontents
\listoftables
\listoffigures
\thispagestyle{empty}
\pagebreak
\setcounter{page}{1}

\section{Abstract}
% The abstract should provide a brief overview of the report.  It should provide
% a summary of the main specific points for the introduction, the main tests and
% experiments, the results, and the conclusions. It is called an abstract because
% you can literally "abstract" sentences from the other sections. 
% 
% Once again, this is not a narrative of your experiences as you executed the
% design.  The abstract should mirror (albeit in a very condensed way) the
% content of your report.

\section{Introduction}
% Brief introduction and overview of the purpose of the lab and of the methods
% and tools used.

\section{Discussion of the Lab}

% This section should include the following:

\subsection{Design Specification}

% In this subsection you will textually describe your client's requirements.
% What does he or she need in the project you are developing.  If you are
% incorporating extra features or capabilities, please describe them clearly in
% this section.

A reference to Table~\ref{table:ex} and one to the Design Specification, Section~\ref{sec:designSpec}.

\begin{itemize}
	\item Overall summary description of the module - 2-3 paragraphs maximum
		(explanation of use cases goes here)

		\begin{itemize}
			\item Specification of the public interface to the module

				\begin{itemize}
					\item Inputs
					\item Outputs
					\item Side effects
				\end{itemize}

			\item Psuedo English description of algorithms, functions, or procedures
			\item Timing constraints
			\item Error handling
		\end{itemize}
\end{itemize}

\begin{table}[h]
	\begin{tabular}{llll}
		\textbf{Item}         & \textbf{Cost} & \textbf{Quantity} & \textbf{Total Cost} \\
		TI RFID Tags          & 0.91          & 20                & 18.20               \\
		RFID Reader/Writer    & 50            & 3                 & 150                 \\
		Xbee Wireless Chips   & 30            & 2                 & 60                  \\
		PLA Makerbot Filament & 48            & 1                 & 48                  \\
		GAL22V10D             & 3.5           & 4                 & 14                  \\
		PICKit 3              & 45            & 2                 & 90                  \\
		SRAM                  & 4             & 2                 & 8                   \\
		3.3 Voltage Regulator & 3.22          & 2                 & 6.44                \\
		Lever Switches        & 2.5           & 6                 & 15                  \\
		16-key Numeric Keypad & 7.5           & 2                 & 15                  \\
		128x169 Color LCD     & 17            & 2                 & 34                  \\
		PIC Microcontrollers  & 7.7           & 4                 & 30.8                \\
		RGB Common Cathode    & 4             & 8                 & 12                  \\
		(EXTRA)               &               &                   &                     \\
		(EXTRA)               &               &                   &                     \\
		(EXTRA)               &               &                   &                    
	\end{tabular}
\end{table}

\subsection{Design Specification\label{sec:designSpec}} 

\subsubsection{Design Requirement\label{sec:requirements}} % Patrick , additions by Alyanna, Ryan

\subsubsection{Identified Use Cases\label{sec:identifiedUseCases}} % Alyanna

\subsubsection{Detailed Specifications\label{detailedSpec}} % Ryan

\subsubsection{Functional Decomposition\label{functions}} % Patrick

\subsection{Hardware Implementation\label{hwImplementation}} 

\subsubsection{Top Level Design\label{hwTopLevel}} % Patrick

\subsubsection{Low Level Design\label{hwLowLevel}} % Patrick

\subsection{Software Implementation\label{swImplementation}}
%
%What is your design????
%
%Present your design starting from a top level functional view and potentially
%block diagram or high level architecture.  Refine that view to present and
%explain each of the modules that comprise the major functional blocks.  Discuss
%the flow of control through the design.  Identify and discuss the specific
%processes/tasks you have implemented in your design. Explain your design
%choices.    

\subsubsection{Top Level Design\label{swTopLevel}} % Alyanna
%
%Put stuff here about the functional decomposition, system architecture,
%interaction of parts.

\subsubsection{Low Level Design\label{swLowLevel}} % Alyanna
%
%task level implementation details here. Control diagram goes here, etc.

\section{Presentation, Discussion, and Analysis of the Results}
%
%Based upon the execution of your design, present your results. Explain them and
%what was expected, and draw any conclusions (for example, did this prove your
%design worked).
%
%In addition to a detailed discussion and analysis of your project and your
%results, you must include all the answers to all questions raised in the lab.
\subsection{Results } % Ryan

\subsection{Discussion of Results } % Alyanna

\subsection{Analysis of Any Errors } % Ryan
%
%This one is obvious. Do this section as appropriate.  If it improves the flow,
%it does not need to be a separate section and may be included in the
%presentation, discussion, and analysis of the results.  However, it will still
%be graded separately and must be present.

\subsection{Analysis of Implementation Issues and Workarounds} % Patrick
%
%State any problems you encountered while working on the project. If your
%project did not work or worked only partially, provide an analysis of why and
%what efforts were made to identify the root cause of any problems. \\
%

\section{Test Plan } % Ryan
%
%Overall summary of what needs to be tested to ensure that your design meets the
%original requirements, 2-3 paragraphs maximum unless specified otherwise

\subsection{Test Specification} % Patrick
%
%Annotated description of what is to be tested and the test limits.  This
%specification quantifies inputs, outputs, and constraints on the system.  That
%is, it provides specific values for each. 
%
%Note, this does not specify test implementation...this is what to do, not how
%to do it.

\subsection{Test Cases} % Alyanna
%
%Annotated description of how your system is to be tested against the test
%limits
%Note, this does specify test implementation...this is not what to do, this is
%how to do it based upon the test specification.

\section{Summary and Conclusion}
%
%You should know these sections very well, no need to explain.  Note, however,
%that they are two different sections.  The summary is just that, a summary of
%your project.  It should loosely mirror the abstract with a bit more detail.
%The conclusion concludes the report, potentially adds information that is often
%outside the main thrust of the report, and may offer suggestions or
%recommendations about the project.

\subsection{Final Summary}


\subsection{Project Conclusions} % Patrick, Alyanna, Ryan
% Include comments and reflections on the capstone. What went well, what could
% be improved next time, What was enjoyable and/or challenging.


\pagebreak
\appendix


\section{Breakdown of Lab Person-hours (Estimated)}
% Use the '&' to separate columns
\begin{tabular}{|l|*{4}{r|}}
	\hline
	Person & Design Hrs & Code Hrs & Test/Debug Hrs & Documentation Hrs \\ \hline
	Patrick & x & x & x & x  \\ \hline
	Alyanna & x & x & x & x \\ \hline
	Ryan & x & x & x & x  \\ \hline
\end{tabular}

~\\

By initializing/signing above, I attest that I did in fact work the
estimated number of hours stated. I also attest, under penalty of shame,
that the work produced during the lab and contained herein is actually my
own (as far as I know to be true). If special considerations or
dispensations are due others or myself, I have indicated them below.

\pagebreak

\section{Hardware Diagrams\label{appendix:hwDiagrams}}
\begin{figure}[H]
	\centering
	\includegraphics[width=\textwidth]{images/BlockDiagram.png}
	\caption{High level block diagram of the system hardware components}
	\label{fig:highLevelBlockDiagram}
\end{figure}

\begin{figure}[H]
	\centering
	\includegraphics[width=0.8\textwidth]{images/SRAMHardwareBlock.png}
	\caption{Block diagram of the SRAM hardware system}
	\label{fig:sramBlockDiagram}
\end{figure}

\begin{figure}[H]
	\centering
	\includegraphics[height=4in]{images/PinOuts.png}
	\caption{Pinouts to the front- and back-end microcontrollers}
	\label{fig:pinoutDiagram}
\end{figure}

\section{Functional Decomposition Diagram\label{fig:funcDecomp}}

\begin{figure}[H]
	\centering
	\includegraphics[width=\textwidth]{images/funDecomp.png}
	\caption{Software functional decomposition showing the major functional divisions and tasks}
	\label{fig:funDecomp}
\end{figure}

\section{State Diagrams}
\subsection{System State Diagram\label{fig:sysStateDiagram}}
\begin{figure}[H]
	\centering
	\begin{subfigure}{0.48\textwidth}
		\includegraphics[width=\textwidth]{images/overallSystemDiagram.png}
		\caption{}
		\label{fig:sysStates}
	\end{subfigure}
	~
	\begin{subfigure}{0.48\textwidth}
		\includegraphics[width=\textwidth]{images/OSStates.png}
		\caption{}
		\label{fig:statesLegend}
	\end{subfigure}
	\caption{State diagram of the primary operating system.
		Figure~\ref{fig:sysStates} shows the states while
		Figure~\ref{fig:statesLegend} provides a legend}
	\label{fig:systemStateDiagram}
\end{figure}

\subsection{General Gameplay State Diagram\label{appendix:gameplayDiagram}}
\begin{figure}[H]
	\centering
	\includegraphics[width=\textwidth]{images/gamePlayState.png}
	\caption{State diagram of basic turn-based game}
	\label{fig:gamePlayState}
\end{figure}

\section{Control Flow Diagrams}

\begin{figure}[H]
	\centering
	\includegraphics[width=0.75\textwidth]{images/keypadDataFlow.png}
	\caption{Keypad response}
	\label{fig:keypadDataFlow}
\end{figure}

\begin{figure}[H]
	\centering
	\includegraphics[width=0.8\textwidth]{images/reactionDataFlow.png}
	\caption{Card Reaction LEDs}
	\label{fig:cardReactionDataFlow}
\end{figure}

\begin{figure}[H]
	\centering
	\includegraphics[width=0.8\textwidth]{images/RFIDreaderDataFlow.png}
	\caption{RFID tag reader subsystem}
	\label{fig:rfidDataFlow}
\end{figure}

\begin{figure}[H]
	\centering
	\begin{subfigure}{\textwidth}
	\includegraphics[width=\textwidth]{images/I2CReceiveDataCtrlFlow.png}
	\caption{}
	\label{fig:i2cReceiveFlow}
	\end{subfigure}

	\begin{subfigure}{\textwidth}
		\includegraphics[width=\textwidth]{images/I2CSendCtrlFlow.png}
		\caption{}
		\label{fig:i2cSendFlow}
	\end{subfigure}
	\caption{I2C communication between microcontrollers. Figure~\ref{fig:i2cReceiveFlow} shows the flow for received data and Figure~\ref{fig:i2cSendFlow} shows the flow for transmitted data.}
	\label{fig:i2cCtrlFlow}
\end{figure}

\section{Project Schedule\label{appendix:ganttChart}}
\begin{figure}[H]
	\centering
	\begin{adjustbox} {rotate=90,center}
		\includegraphics[width=\textwidth]{images/NewGantt.pdf}
	\end{adjustbox}
	\caption{Project Gantt Chart}
	\label{fig:ganttChart}
\end{figure}

\clearpage
\section{Source Code}
% We will put code here. Use the format:
%    \subsection{Main Function}
%    \lstinputlisting{../code/main.c}
%
%    \subsection{Tasks}
%    \subsubsection{Task Control Blocks}
%    \lstinputlisting{../code/tcb.h}
Source code for this project is provided below.

\subsection{System Scheduler}
The system-wide global definitions and variables are also given here for convenience.
\lstinputlisting{../FrontEnd.X/globals.h}
\lstinputlisting{../FrontEnd.X/interrupts.h}
\lstinputlisting{../FrontEnd.X/SchedMain.c}

\subsection{Creating Cards}
\lstinputlisting{../FrontEnd.X/buildCard.c}

\subsection{Example Game}
Both a single player and multiplayer version of the same game are provided.
\lstinputlisting{../FrontEnd.X/game.h}
\lstinputlisting{../FrontEnd.X/game.c}

\subsection{I$^2$C InterPIC Communication}
\lstinputlisting{../FrontEnd.X/i2cComm.h}
\lstinputlisting{../FrontEnd.X/i2cComm.c}

\subsection{Keypad Driver}
\lstinputlisting{../FrontEnd.X/keypadDriver.h}
\lstinputlisting{../FrontEnd.X/keypadDriver.c}

\subsection{LCD Driver}
\lstinputlisting{../FrontEnd.X/LCD.h}
\lstinputlisting{../FrontEnd.X/LCD.c}

\subsection{Card Reaction Control}
\lstinputlisting{../FrontEnd.X/LED.h}
\lstinputlisting{../FrontEnd.X/LED.c}

\subsection{Motor Driver}
\textit{Note: This feature was not implemented in the final product}
\lstinputlisting{../FrontEnd.X/motorDriver.h}
\lstinputlisting{../FrontEnd.X/motorDriver.c}

\subsection{RFID Reader Driver}
\lstinputlisting{../FrontEnd.X/rfidReader.h}
\lstinputlisting{../FrontEnd.X/rfidReader.c}

\subsection{EIA-232 Serial Connection}
\lstinputlisting{../FrontEnd.X/rs232.h}
\lstinputlisting{../FrontEnd.X/rs232.c}

\subsection{SRAM Primary Memory}
\textit{Note: There are two different files for each microcontroller due to hardware configurations}
\lstinputlisting{../FrontEnd.X/SRAM.h}
\lstinputlisting{../FrontEnd.X/SRAMfront.c}
\lstinputlisting{../FrontEnd.X/SRAMback.c}

\subsection{SPI Initialization}
\textit{Note: SPI used for communication with the LCD display}
\lstinputlisting{../FrontEnd.X/startup.h}
\lstinputlisting{../FrontEnd.X/startup.c}

\subsection{Wireless Connectivity via Xbee}
\lstinputlisting{../FrontEnd.X/xbee.h}
\lstinputlisting{../FrontEnd.X/xbee.c}

\end{document}
