% !TEX TS-program = pdflatex
% !TEX encoding = UTF-8 Unicode

% This is a simple template for a LaTeX document using the "article" class.
% See "book", "report", "letter" for other types of document.

\documentclass[12pt]{article} % use larger type; default would be 10pt
\usepackage[utf8]{inputenc}   % set input encoding (not needed with XeLaTeX)

%%% PAGE DIMENSIONS
\usepackage{geometry}
\geometry{a4paper}
\geometry{margin=1in} % 1in page margin

%%% COLOR AND GRAPHICS
\usepackage{color}
\usepackage{graphicx} % support the \includegraphics command and options

\usepackage{pslatex}
\definecolor{mygreen}{rgb}{0,0.6,0}
\definecolor{mygray}{rgb}{0.5,0.5,0.5}
\definecolor{mymauve}{rgb}{0.58,0,0.82}
\usepackage{listings} % For displaying source code
\lstset{ %
  language=C,                      % the language of the code
  backgroundcolor=\color{white},   % choose the background color; you must add \usepackage{color} or \usepackage{xcolor}
  basicstyle=\sffamily\fontsize{11}{13.2}\selectfont,        % the size of the fonts that are used for the code
  breakatwhitespace=false,         % sets if automatic breaks should only happen at whitespace
  breaklines=true,                 % sets automatic line breaking
  captionpos=t,                    % sets the caption-position to bottom
  commentstyle=\color{mygreen},    % comment style
  deletekeywords={...},            % if you want to delete keywords from the given language
  escapeinside={\%*}{*)},          % if you want to add LaTeX within your code
  extendedchars=true,              % lets you use non-ASCII characters; for 8-bits encodings only, does not work with UTF-8
  frame=single,                    % adds a frame around the code
  keepspaces=true,                 % keeps spaces in text, useful for keeping indentation of code (possibly needs columns=flexible)
  keywordstyle=\color{blue},       % keyword style
  morekeywords={*,...},            % if you want to add more keywords to the set
  numbers=left,                    % where to put the line-numbers; possible values are (none, left, right)
  numbersep=5pt,                   % how far the line-numbers are from the code
  numberstyle=\color{mygray},      % the style that is used for the line-numbers
  rulecolor=\color{black},         % if not set, the frame-color may be changed on line-breaks within not-black text (e.g. comments (green here))
  showspaces=false,                % show spaces everywhere adding particular underscores; it overrides 'showstringspaces'
  showstringspaces=false,          % underline spaces within strings only
  showtabs=false,                  % show tabs within strings adding particular underscores
  stepnumber=1,                    % the step between two line-numbers. If it's 1, each line will be numbered
  stringstyle=\color{mymauve},     % string literal style
  tabsize=2,                       % sets default tabsize to 2 spaces
  title=\lstname                   % show the filename of files included with \lstinputlisting; also try caption instead of title
}

% \usepackage[parfill]{parskip} % Activate to begin paragraphs with an empty line rather than an indent

%%% PACKAGES
\usepackage{booktabs} % for much better looking tables
\usepackage{array}    % for better arrays (eg matrices) in maths
\usepackage{paralist} % very flexible & customisable lists (eg. enumerate/itemize, etc.)
\usepackage{verbatim} % adds environment for commenting out blocks of text & for better verbatim
\usepackage{subfig}   % make it possible to include more than one captioned figure/table in a single float

%%% HEADERS & FOOTERS
%\usepackage{fancyhdr} % This should be set AFTER setting up the page geometry
%\pagestyle{fancy} % options: empty , plain , fancy
%\renewcommand{\headrulewidth}{0pt} % customise the layout...
%\lhead{}\chead{}\rhead{}
%\lfoot{}\cfoot{\thepage}\rfoot{}


%%% SECTION TITLE APPEARANCE
\usepackage{sectsty}
\sectionfont{\normalsize\bfseries\uppercase}
\subsectionfont{\normalsize\bfseries}
\subsubsectionfont{\normalsize\mdseries\itshape}

%%% ToC (table of contents) APPEARANCE
\usepackage[nottoc,notlof,notlot]{tocbibind} % Put the bibliography in the ToC
\usepackage[titles,subfigure]{tocloft} % Alter the style of the Table of Contents
\renewcommand{\cftsecfont}{\rmfamily\mdseries\upshape}
\renewcommand{\cftsecpagefont}{\rmfamily\mdseries\upshape} % No bold!

%%% Title setup
\newcommand{\TitleFont}{\fontsize{16}{20}\selectfont\bfseries}
\newcommand{\AuthorFont}{\fontsize{14}{17}\selectfont}

%%% END Article customizations

%%% The "real" document content comes below...

\title{\TitleFont EE 478 Capstone Final Report \\ RFID Interaction Suite \vfill }
\author{\AuthorFont Alyanna Castillo \\ Patrick Ma \\ Ryan McDaniels}
\date{}

\begin{document}

%% Make title and ToC, start page numbering AFTER ToC
\maketitle
\thispagestyle{empty}
\pagebreak
\tableofcontents
\listoftables
\listoffigures
\thispagestyle{empty}
\pagebreak
\setcounter{page}{1}

\section{Abstract}
% The abstract should provide a brief overview of the report.  It should provide
% a summary of the main specific points for the introduction, the main tests and
% experiments, the results, and the conclusions. It is called an abstract because
% you can literally "abstract" sentences from the other sections. 
% 
% Once again, this is not a narrative of your experiences as you executed the
% design.  The abstract should mirror (albeit in a very condensed way) the
% content of your report.
 
\section{Introduction}
% Brief introduction and overview of the purpose of the lab and of the methods
% and tools used.

\section{Discussion of the Lab}

% This section should include the following:

\subsection{Design Specification}

% In this subsection you will textually describe your client's requirements.
% What does he or she need in the project you are developing.  If you are
% incorporating extra features or capabilities, please describe them clearly in
% this section.

\begin{itemize}[$$]
	\item Overall summary description of the module - 2-3 paragraphs maximum
		(explanation of use cases goes here)

		\begin{itemize}[$$]
			\item Specification of the public interface to the module

				\begin{itemize}[$\bullet$]
					\item Inputs
					\item Outputs
					\item Side effects
				\end{itemize}

			\item Psuedo English description of algorithms, functions, or procedures
			\item Timing constraints
			\item Error handling
		\end{itemize}
\end{itemize}

\begin{table}[h]
		\centering
		\label{table:ex}
		\begin{tabular}{llr}
				\toprule
				Animal    & Description & Price (\$) \\
				\midrule
				Gnat      & per gram    & 13.65      \\
									& each        & 0.01       \\
				Gnu       & stuffed     & 92.50      \\
				Emu       & stuffed     & 33.33      \\
				Armadillo & frozen      & 8.99       \\
				\bottomrule
		\end{tabular}
		\caption{Example table.}
\end{table}

\subsection{Software Implementation }
%
%What is your design????
%
%Present your design starting from a top level functional view and potentially
%block diagram or high level architecture.  Refine that view to present and
%explain each of the modules that comprise the major functional blocks.  Discuss
%the flow of control through the design.  Identify and discuss the specific
%processes/tasks you have implemented in your design. Explain your design
%choices.    

\subsubsection{Top level design }
%
%Put stuff here about the functional decomposition, system architecture,
%interaction of parts.

\subsubsection{low level design  }
%
%task level implementation details here. Control diagram goes here, etc.

\section{Presentation, Discussion, and Analysis of the Results}
%
%Based upon the execution of your design, present your results. Explain them and
%what was expected, and draw any conclusions (for example, did this prove your
%design worked).
%
%In addition to a detailed discussion and analysis of your project and your
%results, you must include all the answers to all questions raised in the lab.
\subsection{results }

\subsection{discussion of results }

\subsection{Analysis of any Errors }
%
%This one is obvious. Do this section as appropriate.  If it improves the flow,
%it does not need to be a separate section and may be included in the
%presentation, discussion, and analysis of the results.  However, it will still
%be graded separately and must be present.

\subsection{Analysis of problems and issues encountered and what efforts were
made to identify the root cause of any problems  }
%
%State any problems you encountered while working on the project. If your
%project did not work or worked only partially, provide an analysis of why and
%what efforts were made to identify the root cause of any problems. \\
%

\section{Test Plan }
%
%Overall summary of what needs to be tested to ensure that your design meets the
%original requirements, 2-3 paragraphs maximum unless specified otherwise

\subsection{Test Specification}
%
%Annotated description of what is to be tested and the test limits.  This
%specification quantifies inputs, outputs, and constraints on the system.  That
%is, it provides specific values for each. 
%
%Note, this does not specify test implementation...this is what to do, not how
%to do it.

\subsection{Test Cases  }
%
%Annotated description of how your system is to be tested against the test
%limits
%Note, this does specify test implementation...this is not what to do, this is
%how to do it based upon the test specification.

\section{Summary and Conclusion}
%
%You should know these sections very well, no need to explain.  Note, however,
%that they are two different sections.  The summary is just that, a summary of
%your project.  It should loosely mirror the abstract with a bit more detail.
%The conclusion concludes the report, potentially adds information that is often
%outside the main thrust of the report, and may offer suggestions or
%recommendations about the project.

\subsection{Final Summary}


\subsection{Project Conclusions}


\pagebreak
\appendix

\section{Source Code}

Source code for this project is provided below.

\subsection{The first part}
% \lstinputlisting{source.c}

\subsection{The second part}
% \lstinputlisting{source.c}

\end{document}
